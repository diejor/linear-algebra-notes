
% basics
\usepackage[english]{babel}
\usepackage{geometry}
\usepackage[shortlabels]{enumitem}
\usepackage{marginnote}
\usepackage{xcolor}


\usepackage{amsmath, amsfonts, mathtools, amsthm, amssymb}
\newcommand\abs[1]{\left|#1\right|}
\newcommand\floor[1]{\lfloor#1\rfloor}
\newcommand\ceil[1]{\lceil#1\rceil}

% theorems
\usepackage{thmtools}
\usepackage{mdframed}

% define a box for definitions
\declaretheoremstyle[
    mdframed={
        backgroundcolor=gray!10,
        hidealllines=true,
        innertopmargin=6pt,
        innerbottommargin=6pt,
        skipabove=\topsep,
        skipbelow=\topsep
    }
]{graybox}

\declaretheorem[
    name=Theorem,
    style=graybox
]{theorem}

\declaretheorem[
    name=Corollary,
    style=graybox
]{corollary}

\declaretheorem[
    name=Definition,
    style=graybox
]{definition}

\declaretheorem[
    name=Algorithm,
    style=graybox
]{algorithm}

\declaretheorem[name=Exercise,style=graybox]{exercise}

\newenvironment{solution}{}{}
\newenvironment{enumerateproblem}
{
    \begin{enumerate}[(a)]
}{
    \end{enumerate}
}

\newcommand{\resetcounters}{
    \setcounter{theorem}{0}
    \setcounter{corollary}{0}
    \setcounter{definition}{0}
    \setcounter{algorithm}{0}
    \setcounter{exercise}{0}
}